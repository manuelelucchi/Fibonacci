\documentclass[amsmath,amssymb,aps,pra,reprint,groupedaddress,showpacs]{revtex4-1}

\usepackage{multirow}
\usepackage{verbatim}
\usepackage{color,graphicx, verbatim, float}
 
\newcommand{\Proof}{\noindent\textbf{Proof.}\quad}
\newcommand{\qed}{\hfill$\Box$}
\newcommand{\ket}[1]{\left\vert #1 \right\rangle}
\newcommand{\bra}[1]{\left\langle #1 \right\vert}
\newcommand{\inpro}[2]{\left\langle #1 \vert #2 \right\rangle}

\newtheorem{theorem}{Theorem}
\newtheorem{lemma}[theorem]{Lemma}
\newtheorem{corollary}[theorem]{Corollary}
\newtheorem{proposition}[theorem]{Proposition}

\newtheorem{innercustomthm}{Theorem}
\newenvironment{customthm}[1]
  {\renewcommand\theinnercustomthm{#1}\innercustomthm}
  {\endinnercustomthm}
  
\newtheorem{innercustomlemma}{Lemma}
\newenvironment{customlemma}[1]
  {\renewcommand\theinnercustomlemma{#1}\innercustomlemma}
  {\endinnercustomlemma}
  
\newtheorem{innercustomproposition}{Proposition}
\newenvironment{customproposition}[1]
  {\renewcommand\theinnercustomproposition{#1}\innercustomproposition}
  {\endinnercustomproposition}

% AUTHOR
\newcommand\Author{{\author{Lucchi Manuele}
\email[]{manuele.lucchi@studenti.unimi.it}
\affiliation{IT Department, Universita' degli Studi di Milano, Citta' degli Studi, Milano, Italia}}}

% ALL BOLD
\newcommand\AllBold[1]{{\boldmath\textbf{#1}}}

\begin{document} 

%%%%%%%%%%%%%%% TITLE %%%%%%%%%%%%%%%%
\title{Fibonacci Sequence Prime Factors Properties}
 
%%%%%%%%%%%%%%% AUTHORS %%%%%%%%%%%%%%%%
\author{Lucchi Manuele}
\email[]{manuele.lucchi@studenti.unimi.it}
\affiliation{IT Department Students, Universita' degli Studi di Milano, Citta' degli Studi, Milano, Italia}

%%%%%%%%%%%%%%% DATE %%%%%%%%%%%%%%%%
\date{\today}

%%%%%%%%%%%%%%% ABSTRACT %%%%%%%%%%%%%%%%
\begin{abstract}
The purpose of this research is to find a common pattern in the scomposition at prime factors of the Fibonacci Numbers. 
This would allow to decompose these numbers avoiding a high computational cost. > <
\end{abstract} 
 
\maketitle

%%%%%%%%%%%%%%% INTRODUCTION %%%%%%%%%%%%%%%%
\section{Introduction}

The Fibonacci Sequence [1] is a recurrent equation introduced by Fibonacci in 1202. This sequence rapidly increases the size of its numbers and has a lot of properties.
The sequence is $F_n = F_{n-1} + F_{n-2}$. %ALTRO?

\begin{figure}[H]
\centering
\includegraphics[width=3in]{assets/fibonacci(100).png}
\caption{The Fibonacci Sequence}
\end{figure} 

%%%%%%%%%%%%%%% POWER OF 2 BEHAVIOURS IN FIBONACCI NUMBERS %%%%%%%%%%%%%%
\section{Powers of 2 behaviours in Fibonacci Numbers}

We first tested with the simpler prime numbers. If we look at the power of 2 present in the first 100 Fibonacci Numbers we get something like this

\begin{figure}[H]
\centering
\includegraphics[width=3in]{assets/factor(100,2).png} % Power of 2 in 100
\caption{Power of 2 as factor of the first 100 Fibonacci Numbers}
\end{figure}

It's pretty easy to see some sort of regularity between the power of 2 included in each Fibonacci Number. We can now increase the range to 1000 to see it better

\begin{figure}[H]
\centering
\includegraphics[width=3in]{assets/factor(1000,2).png} % Power of 2 in 1000
\caption{Power of 2 as factor of the first 1000 Fibonacci Numbers}
\end{figure}

And again we can see a similar behaviour. First of all, we can notice how only numbers that are multiple of 3 are divisible by 2. 
  
So we can suppose that every number divisible by 3 has its Fibonacci equivalent to be even

\begin{figure}[H]
\centering
\includegraphics[width=3in]{assets/factor(100,2)only(3).png} % Power of 2 in 100 divisible for 3
\caption{Power of 2 as factor of the first 100 Fibonacci Numbers but only for $n$ divisible by 3}
\end{figure}

Instead, if we look only at every $n$ not divisible by 3, we get something like this

\begin{figure}[H]
\centering
\includegraphics[width=3in]{assets/factor(100,2)except(3).png} % Power of 2 in 100 not divisible for 3
\caption{Power of 2 as factor of the first 100 Fibonacci Numbers but only for $n$ not divisible by 3}
\end{figure}

So we can draw a first conclusion: 
$$\mathbf{\forall n \in \mathbb{N} : 3\mid n \implies 2\mid Fibonacci(n)}$$

Going forward, we can define the regularity for every exponent of 2, 
for example, $F(n)$ is divisble by 2 starting from 3 (or $2^0 \cdot 3$) and $\forall n \in \mathbb{N}$ that is a multiple of 3.
If we search for Fibonacci Numbers divisible for $2^3$, we have to start from 6 (or $2^1 \cdot 3$)
and increase of 12 (or $2^2 \cdot 3$) at time. For $2^4$ the starting $n$ becomes 12 (or $2^2 * 4$)
and the increase between similar exponents is 24 (or $2^4 \cdot 3$).\\

So, after defining E as a certain exponent of 2 $\geq 3$ inside Fibonacci Numbers we can set the first $n$ which will appear as $2^{E-1}$ and the difference between the $n_s$ that have the same exponent of 2 in $F(n)$
as $2^E \cdot 3$.\\

However, the regularity in the exponents of 2 is not $\forall n$, since 0, 1, and 2 don't follow it,
otherwise, with 1 as exponent we would have a 3 as a starting point, but we got 6. Also, it seems like that 2 as an exponent will not appear,
so \AllBold{we will not find any $F(n)$ that can be divided by $2^2$ and not for $2^k$ with $k \in N \land k>2$ at the same time}

%%%%%%%%%%%%%%% A FIRST SPECIFIC FORMULA %%%%%%%%%%%%%%%%
\section{A first, specific Formula}

We have now all the tools for the development of a simple equation that gives us the power of in the $n_{th}$ number of Fibonacci. Let's define $S_n$ as the exponent of 2 of the $n_{th}$ Fibonacci Number, 
we can say that $\mathbf{S_n = S_{n/2} + 1}$. But that is true for $F(n)$ only where $n$ is divisible by both 2 and 3, so we have to create special cases for
$n=1$ and divisible by 3 or 2 but not both. For $n = 1$ or $n \nmid 2 \land \mid 3 $ it's always 0, for $2 \mid n \land 3 \nmid n $ it's always one.

So the final recurrent formula will be: 
$$S_n = \begin{cases} S_{n/2} + 1, & \mbox{if }2|n \land 3 \mid n \\ 1, & \mbox{if } 2 \nmid n \land 3|n \\ 0, & \mbox{if } \left( 2 \mid n \land 3 \nmid n \right) \lor n = 1 \lor n = 0
\end{cases} $$

%//DECISAMENTE MIGLIORAMENTO FORMULA

%%%%%%%%%%%%%%% A RECURRENT BEHAVIOR IN PRIME FACTORS SCOMPOSITION OF FIBONACCI NUMBERS %%%%%%%%%%%%%%%%
\section{A recurrent behavior in Prime Factors Scomposition of Fibonacci Numbers}

2 is not the only base to present a similar behavior. Every number (even non prime ones) have a similar regularity, but with some differences.
For example the 3

\begin{figure}[H]
\centering
\includegraphics[width=3in]{assets/factor(100,3).png} % Power of 3 in 100
\caption{Power of 3 as factor of the first 100 Fibonacci Numbers}
\end{figure}

Or the 5

\begin{figure}[H]
\centering
\includegraphics[width=3in]{assets/factor(100,5).png} % Power of 5 in 100
\caption{Power of 5 as factor of the first 100 Fibonacci Numbers}
\end{figure}

And so on even for bigger prime numbers like 23

\begin{figure}[H]
\centering
\includegraphics[width=3in]{assets/factor(10000,23).png} % Power of 23 in 10000
\caption{Power of 23 as factor of the first 10000 Fibonacci Numbers}
\end{figure}

%//INIZIO FOCUS STARTING POINT

\begin{comment}

But it's not exactly the same, it seems like the more we increase the base, the less the number appears as a prime factor in Fibonacci Numbers 
(so with an increasing distance between the same exponents), but also this is not true, there are a lot of cases that don't follow the growing rule.

But before getting in a more generic formula, we have to name some things (since they are going to be different for every base).

\begin{enumerate} 
\item $\mathbf{\Gamma}$ will be the base of the prime (or non-prime) number we are taking as example. 
\item $\mathbf{\Omega}$ will be the exponent of base $\Gamma$ of $F(n)$ 
\item $\mathbf{\Delta \Omega}$ will be defined as the distance between the Fibonacci Number before the next equal exponent will appear
\item $\mathbf{F(n)}$ will be the $n_{th}$ Fibonacci Number
\item $\mathbf{\lambda(\Omega, \Gamma)}$ will identify the function that returns the first $n$ where the exponent of the $\Gamma$ base appears in $F(n)$
\item $\mathbf{\lambda_n(\Omega, \Gamma)}$ will, in the same way, be the $n_{th}$ Fibonacci Number where the given exponent for the given base appears
\end{enumerate} 

If we look at FIG. 6 and 7, we have two examples to notice the differences between different bases.
For $\Gamma = 3$ and $\Gamma = 5$ we have $\lambda(1, 3) = 4$ and $\lambda(1, 5) = 6$.
It seems like $\lambda(1, \Gamma) = \Gamma + 1$, but that's wrong: for example, 
for $\Gamma = 11$ we have $\lambda(1, 11) = 10$. For $\Gamma = 13$ it will be $n = 7$. 
So let's have a look at the starting points of the first prime numbers:

\end{comment}

%//FINE FOCUS STARTING POINT

%%%%%%%%%%%%%%% FOCUS ON NON PRIME FACTORS %%%%%%%%%%%%%%%%
\section{Focus on non-prime factors}

Until now we tested only prime numbers, but we can expect a similar behavior for non-primes.
For example if we look at power of 4 $(2^2)$ we notice that it's exactly halved compared of 2. 

\begin{figure}[H]
\centering
\includegraphics[width=3in]{assets/factor(100,[2,4]).png}
\caption{Comparison of powers of 2 and 4 of the first 1000 Fibonacci Numbers}
\end{figure}

We can have better overview by comparing the distance between the same exponent of two different factors 
as they first appear. For example, using 2 and 4 as factors and comparing the index in Fibonacci numbers
when they are on 1 by subtracting we get this:

\begin{figure}[H]
\centering
\includegraphics[width=3in]{assets/distance(1000,2,4,1,minus).png}
\caption{Difference between indexes of the same recurrence of the same grade of 2 and 4}
\end{figure}

As expected it's a linear function. If we try to divide them, again we can see the ratio it's exactly $\frac{1}{2}$

\begin{figure}[H]
\centering
\includegraphics[width=3in]{assets/distance(1000,2,4,1,div).png}
\caption{Division between indexes of the same recurrence of the same grade of 2 and 4}
\end{figure}

%%%%%%%%%%%%%%% COMPARISON ON DIFFERENT BASES %%%%%%%%%%%%%%%%
\section{Comparison on different bases}

But if we try to obtain a choerent behaviour for other couple of numbers we'll be disappointed. The examples below are between 9 and 3

\begin{figure}[H]
\centering
\includegraphics[width=3in]{assets/distance(1000,3,9,1,minus).png}
\caption{Difference between indexes of the same recurrence of the same grade of 3 and 9}
\end{figure}

\begin{figure}[H]
\centering
\includegraphics[width=3in]{assets/distance(1000,3,9,1,div).png}
\caption{Division between indexes of the same recurrence of the same grade of 3 and 9}
\end{figure}

While in both of them we can see some sort of logic, it's not $1/3$ nor the square root ($9 = 3^2$).
However, this sort of behavior is once again recurrent for every number, even between numbers that doesn't have anything in common.
For example, 2 and 3:

\begin{figure}[H]
\centering
\includegraphics[width=3in]{assets/distance(500,2,3,1,minus).png}
\caption{Difference between indexes of the same recurrence of the same grade of 2 and 3}
\end{figure}

\begin{figure}[H]
\centering
\includegraphics[width=3in]{assets/distance(500,2,3,1,div).png}
\caption{Division between indexes of the same recurrence of the same grade of 2 and 3}
\end{figure}

We tested a lot of possibilities but couldn't be able to find what connect them all, except one thing:
In every test made by dividing the corrispondent indexes, going forward with the iterations shows the ratio to stabilize near a certain number
We can see that already in FIG. 13 where it goes somewhere between 0.44 and 0.45 or in FIG.15 where it tends to 1.
On the other hand, subtracting the indexes shows some different behaviours, from linear functions to regular curves that tries to mimic linear functions.
Or again, what we saw in FIG. 14, that' pretty regular but different from everything else seen until now.
Of course the same test can be done with different exponent grades with the same properties.

%%%%%%%%%%%%%%% CONCLUSION %%%%%%%%%%%%%%%%
\section{Conclusion}

As we know, the Fibonacci Numbers hold a great number of interesting properties and this one sums up to the list.

%%%%%%%%%%%%%%% REFERENCES %%%%%%%%%%%%%%%%
\begin{thebibliography}{24}
 
\bibitem{fibonaccigeneral}
{OEIS},
\textit{Sequence A000045 (Fibonacci Numbers)}

\end{thebibliography}

\end{document}

%%https://www.overleaf.com/learn/latex/Positioning_images_and_tables
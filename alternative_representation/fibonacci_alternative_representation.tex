\documentclass[amsmath,amssymb,aps,pra,reprint,groupedaddress,showpacs]{revtex4-1}

\usepackage{multirow}
\usepackage{verbatim}
\usepackage{color,graphicx}

\newcommand{\Proof}{\noindent\textbf{Proof.}\quad}
\newcommand{\qed}{\hfill$\Box$}
\newcommand{\ket}[1]{\left\vert #1 \right\rangle}
\newcommand{\bra}[1]{\left\langle #1 \right\vert}
\newcommand{\inpro}[2]{\left\langle #1 \vert #2 \right\rangle}

\newtheorem{theorem}{Theorem}
\newtheorem{lemma}[theorem]{Lemma}
\newtheorem{corollary}[theorem]{Corollary}
\newtheorem{proposition}[theorem]{Proposition}

\newtheorem{innercustomthm}{Theorem}
\newenvironment{customthm}[1]
  {\renewcommand\theinnercustomthm{#1}\innercustomthm}
  {\endinnercustomthm}
  
\newtheorem{innercustomlemma}{Lemma}
\newenvironment{customlemma}[1]
  {\renewcommand\theinnercustomlemma{#1}\innercustomlemma}
  {\endinnercustomlemma}
  
\newtheorem{innercustomproposition}{Proposition}
\newenvironment{customproposition}[1]
  {\renewcommand\theinnercustomproposition{#1}\innercustomproposition}
  {\endinnercustomproposition}

% AUTHOR
\newcommand\Author{{\author{Lucchi Manuele}
\email[]{manuele.lucchi@studenti.unimi.it}
\affiliation{IT Department, Universita' degli Studi di Milano, Citta' degli Studi, Milano, Italia}}}

% ALL BOLD
\newcommand\AllBold[1]{{\boldmath\textbf{#1}}}

\begin{document} 

\title{Fibonacci Sequence Alternative Representation}
 
\author{Lucchi Manuele}
\email[]{manuele.lucchi@studenti.unimi.it}
\affiliation{IT Department Students, Universita' degli Studi di Milano, Citta' degli Studi, Milano, Italia}

\date{\today}

\begin{abstract}
In this Research we will introduce a new representation of the Fibonacci Sequence starting from its Generative Function, the Binet Formula. 
The result will be an equation that can be parallelized in a multithreaded system and that only uses positive integers.
\end{abstract}

\maketitle

\section{Introduction}
The Fibonacci Sequence [1] is a simple, but really important, Sequence in Natural Numbers that can be found in a lot of places in nature, 
for example flowers' petals follow it in their numeration. The sequence is just the sum of its previous terms: 
$$ F_{n} = F_{n-1} + F_{n-2} $$
In 1843 Binet introduced its Generative Function called the Binet Formula [2] (even if it was already known a century earlier by Euler, Daniel Bernoulli and de Moivre) that allows everyone to
calculate the $n_{th}$ Fibonacci Number without using a recursive approach
$$f(n) = \frac{1}{\sqrt{5}} \left[ \left( \frac{1+\sqrt{5}}{2} \right) ^n - \left( \frac{1-\sqrt{5}}{2} \right)^n \right] $$
However this representation uses irrational numbers like $\sqrt{5}$ that causes a certain level of imprecision in the result that should be a positive integer.
Our approach consists in using the Newton Binomial [3] to simplify all the irrational numbers and then optimize the result.

\section{Transformation}

The first step is to take out the $\frac{1}{2^n}$
$$f(n) = \frac{1}{2^n\sqrt{5}} \left[ \left( 1+\sqrt{5} \right) ^n - \left( 1-\sqrt{5} \right)^n \right]$$
Now we can notice that $(1 + \sqrt{5})^n$ and $(1 - \sqrt{5})^n$ are both in the form $(a + b)^n$ so they are suitable for the Newton Binomial transformation,
so we can write them like \\
$ A_n:= \sum_{k=0}^n \binom{n}{k} \left(\sqrt{5} \right) ^k $ and $ B_n:= \sum_{k=0}^n \binom{n}{k} \left( -\sqrt{5} \right) ^k $ both with $k \in N$.
If we unroll the sums we'll find some similarities between A and B.\\
So
\begin{gather*}
A = a\sqrt{5} +b5 + c\sqrt{5^3} + d5^2 + ... + z\sqrt{ 5 }^k \\
B = -a\sqrt{5} +b5 + -c\sqrt{5^3} + d5^2 + ... + z \left( -\sqrt{5} \right) ^k
\end{gather*} 
with $a, b, c, d,$  ..., $z \in N$. \\

Since it's $f(n)=\frac{1}{2^n\sqrt{5}} \left[ A - B \right]$ we'll have
\begin{align*}
A - B &= a\sqrt{5} +b5 + c\sqrt{5^3} + d5^2 + ... + z\sqrt{5}^k  \\
&- \left( -a\sqrt{5} +b5 + -c\sqrt{5^3} + d5^2 + ... + z \left( -\sqrt{5} \right)^k \right) \\
&= a\sqrt{5} +b5 + c\sqrt{5^3} + d5^2 + ... + z\sqrt{5}^k +a\sqrt{5} -b5 \\
&+ c\sqrt{5^3} - d5^2 + ... - z \left( -\sqrt{5} \right) ^k
\end{align*}

Now we can see that all members that have an integer exponent can be reduced. If we define $C:= A - B$ we get
\begin{align*}
C_n &= \sum_{k=0}^n \binom{n}{k} \left(\sqrt{5} \right) ^k - \sum_{k=0}^n \binom{n}{k} \left(-\sqrt{5} \right) ^k\\
&= \sum_{k=0}^n \binom{n}{k} \left[ \left(\sqrt{5} \right) ^k - \left(-\sqrt{5} \right) ^k \right]
\end{align*}
We can now split the sum again, this time differentiating between odd and even steps

\begin{align*}
C_n &= \sum_{k=0}^{n/2} \binom{n}{2k} \left[ \left(\sqrt{5} \right) ^{2k} - \left(-\sqrt{5} \right) ^{2k} \right]\\
&+ \sum_{k=0}^{n/2} \binom{n}{2k + 1} \left[ \left(\sqrt{5} \right) ^{2k + 1} - \left(-\sqrt{5} \right) ^{2k +1} \right]
\end{align*}

Now if we analyze the sums one by one, we can observe that the first one converges to 0, $\forall n \in{N}$ where $n$ is odd
$$ \sum_{k=0}^{\frac{n}{2}} \binom{n}{2k} \left[ \left(\sqrt{5} \right) ^{2k} - \left(-\sqrt{5} \right) ^{2k} \right] = 0 $$
while the second one will become

\begin{align*}
C &= \sum_{k=0}^{\frac{n}{2}} \binom{n}{2k + 1} \left[ \left(\sqrt{5} \right) ^{2k + 1} - \left(-\sqrt{5} \right) ^{2k + 1} \right]\\
&= 2 \sum_{k=0}^\frac{n}{2} \binom{n}{2k + 1} \left(\sqrt{5} \right) ^{2k + 1}
\end{align*}

Since $2k + 1$ is always odd, we know that 

\begin{align*}
&\left(-\sqrt{5} \right) ^{2k + 1} =  - \left( \sqrt{5} \right) ^{2k + 1} \implies \\
&\left( \sqrt{5} \right) ^{2k + 1} - \left(-\sqrt{5} \right) ^{2k + 1} = 2\left(\sqrt{5} \right) ^{2k + 1}
\end{align*}

In the end what remains is our $S_n$
\begin{align*}
S_n &= \frac{2}{2^n\sqrt{5}} \sum_{k=0}^\frac{n}{2} \binom{n}{2k + 1} \left( \sqrt{5} \right) ^{2k +1} \\
&=\frac{1}{2^{n-1}} \sum_{k=0}^\frac{n}{2} \binom{n}{2k + 1}5^{k}
\end{align*}

This is a first version of the alternative formula that already supports a certain level of parallelization and, since every Fibonacci's Number is an integer, doesn't need to handle real numbers like the Binet one

\begin{thebibliography}{24}
 
  \bibitem{fibonaccigeneral}
    {OEIS},
    \textit{Sequence A000045 (Fibonacci Numbers)}

  \bibitem{binetgeneral}
    {DA VEDERE},
    \textit{DA VEDERE}

  \end{thebibliography}

\end{document}